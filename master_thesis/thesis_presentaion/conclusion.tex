\section{Conclusion}

\begin {frame}
\frametitle{General results.}
\begin{itemize}
  \item Derivation of analytical formulas for performance evaluation;
  \item Numerical experiments are carried out;
  \item D2D enabled cellular network becomes useful when the signal propagation is seriously obstructed;
  \item D2D enabled cellular network yields good performance when the ratio of the number of base stations to the number of mobile users is around \(1/10,000\).
\end{itemize}
\end {frame}

\begin{frame}
\frametitle{Future work}
\begin{itemize}
  \item Consideraion of \(n\)--D2D relay links in both non-fading channel and channel with fading;
  \item Interference modelling;
  \item Additional relaying layer can be added into the model and performance evaluation is also possible (like in project 'OneWeb').
\end{itemize}
\end{frame}

\begin{frame}
\frametitle{Acknowledgements}
Student Gulomov Saidkhuja would like to express his deep gratitude to prof. Naoto Miyoshi for his extensive support and help during the preparation of the work.
\end{frame}
