\subsection {Case 3: Connectivity by means of a multiple device-to-device relay links}
This scenario assumes the distance between the receiver and its nearest serving base station is $> R_1+R_2$, thus enabling multiple ($>2$) device-to-device relay links.
\par {\bf Approach}
\\As the nearest serving base station is unreachable for a direct link, the nearest node will always be a mobile user.
Let's denote \(l_{i,j}\) as the distance between a node \(i\) and its nearest neighbour \(j\).
As (9) states, the successful connection between two nearest nodes happens when the distance between them does not exceed the value of \(R_2\).
The \(0\)-th node will always denote the receiver, that is located at the origin of the considered dimension.
Denoting the number of hops as \(n=0,1,2,\dots\), we get:
\[n=1\text{ , }\P[l_{0,1}\le R_2\big|d_0>R_1]\]
\[n=2\text{ , }\P[l_{0,1}\le R_2]\times\P[l_{1,2}\le R_2\big|l_{0,1}\le R_2,d_0>R_1,d_1>R_1]\]
\[n=3\text{ , }\P[l_{0,1}\le R_2]\times\P[l_{1,2}\le R_2\big|l_{0,1}\le R_2]\]
\[\times\P[l_{2,3}\le R_2\big|l_{0,1}\le R_2,l_{1,2}\le R_2,d_0>R_1,d_1>R_1,d_2>R_1]\]
In order to exclude the possibility of the base station being in between any of the nodes, we've added the condition \(d_{i}>R_1\text{, }i=0,1,2,\dots,n-1\), that denotes the distance between the \(i\)-th node and its nearest base station.
\\The general form of the \(n\)-hop connection will take form:
\begin{framed}
\[\sum_{n=0}^{\infty}\Bigg[\prod_{k=1}^{n}\P\Big[l_{k-1,k}\le R_2\Big|\bigcap\limits_{h=1}^{k-1}l_{h-1,h}\le R_2, \bigcap\limits_{m=0}^{n-1}d_m>R_1\Big]\Bigg]\]
\begin{equation}
\times\P\Big[d_{n}\le R_1\Big|\bigcap\limits_{q=1}^{n}l_{q-1,q}\le R_2\text{, }d_{q-1}>R_1\Big]
\end{equation}
\end{framed}
PDF of \(\P[d_m>R_1] (m=0,1,\dots,n-1)\) is:
\[\frac{d}{dr}\P[d_i>r]=\frac{d}{dr}\P[\Phi_b(B_m)=0]\]
In \(\R^1\) case, \(B_m\) is the interval of search for the nearest base station; and in \(\R^2\) case -- the disk of search. Thus,
\begin{equation}
\R^1\text{: }f^c_d(r)=\frac{d}{dr}\exp(-\lambda_br)=-e^{-\lambda_br}
\end{equation}
\begin{equation}
\R^2\text{: }f^c_d(r)=\frac{d}{dr}\exp(-\lambda_b\pi r^2)=-2\lambda_b\pi re^{-\lambda_b\pi r^2}
\end{equation}

\subsection {Special cases of 2 and 3 relay links}
\par In order to provide legitimate numerical results, we shall consider special cases of the number of multiple D2D relay links being 2 and 3.
\par{\bf Two D2D relay links, \(n=2\)}
\[\P[SNR^{2-hop}_{nf}\ge\Theta_b]=\]
\[\P\Bigg[l_{0,1}\le R_2\Bigg|d_0>R_1\Bigg]\]
\[\times\P\Bigg[l_{1,2}\le R_2\Bigg|l_{0,1}\le R_2\text{, }d_0>R_1\text{, }d_1>R_1\Bigg]\]
\[\times\P\Bigg[d_2\le R_1\Bigg|l_{0,1}\le R_2\text{, }l_{1,2}\le R_2\text{, }d_0>R_1\text{, }d_1>R_1\Bigg]\]
\[=\int_{0}^{R_1}\Big(1-\P[l_{0,1}>R_2]\Big)f^c_d(s)ds\]
\[\times\int_{R_2/2}^{R_1+R_2}\int_{0}^{R_1}\int_{R_2/2}^{R_2}\Big(1-\P[l_{1,2}>R_2\Big)f^c_d(s_1,s_2)f_l(t)dtds_1ds_2\]
\[\times\int_{\frac{3}{2}R_2}^{2R_2}\int_{R_2/2}^{R_2}\int_{R_2/2}^{R_1+R_2}\int_{0}^{R_1}\Big(1-\P[d_2>R_1]\Big)f_l(t_1,t_2)f^c_d(s_1,s_2)dt_1dt_2ds_1ds_2\]
\[=-2\lambda_b\pi\int_{0}^{R_1}\Big(1-e^{-\lambda_u\pi R_2^2}\Big)se^{-\lambda_b\pi s^2}ds\]
\[\times\]

\par{\bf Three D2D relay links, \(n=3\)}
\[\P[SNR^{3-hop}_{nf}\ge\Theta_b]\]
\[=\P\Bigg[l_{0,1}\le R_2\Bigg|d_0>R_1\Bigg]\]
\[\times\P\Bigg[l_{1,2}\le R_2\Bigg|l_{0,1}\le R_2\text{, }d_0>R_1\text{, }d_1>R_1\Bigg]\]
\[\times\P\Bigg[l_{2,3}\le R_2\Bigg|l_{0,1}\le R_2\text{, }l_{1,2}\le R_2\text{, }d_0>R_1\text{, }d_1>R_1\text{, }d_2>R_1\Bigg]\]
\[\times\P\Bigg[d_3\le R_1\Bigg|l_{0,1}\le R_2\text{, }l_{1,2}\le R_2\text{, }l_{2,3}\le R_2\text{, }d_0>R_1\text{, }d_1>R_1\text{, }d_2>R_1\Bigg]\]
\[=\int_{0}^{R_1}\Big(1-\P[l_{0,1}>R_2]\Big)f^c_d(s)ds\]
\[\times\int_{R_2/2}^{R_1+R_2}\int_{0}^{R_1}\int_{R_2/2}^{R_2}\Big(1-\P[l_{1,2}>R_2\Big)f^c_d(s_1,s_2)f_l(t)dtds_1ds_2\]
\[\times\int_{\frac{3}{2}R_2}^{2R_2+R_1}\int_{R_2/2}^{R_2+R_1}\int_{0}^{R_1}\int_{\frac{3}{2}R_2}^{2R_2}\int_{R_2/2}^{R_2}\Big(1-\P[l_{2,3}>R_2]\Big)f_l(t_1,t_2)f^c_d(s_1,s_2,s_3)\]
\[\times dt_1dt_2ds_1ds_2ds_3\]
\[\times\int_{\frac{3}{2}R_2}^{2R_2+R_1}\int_{R_2/2}^{R_2+R_1}\int_{0}^{R_1}\int_{\frac{5}{2}R_2}^{3R_2}\int_{\frac{3}{2}R_2}^{2R_2}\int_{R_2/2}^{R_2}\Big(1-\P[d_3>R_1]\Big)f_l(t_1,t_2,t_3)f^c_d(s_1,s_2,s_3)\]
\[\times dt_1dt_2dt_3ds_1ds_2ds_3\]
\[=\]
