\section{Conclusion}
The consideration of SNR in cellular and D2D link cases resulted in a derivation of analytical formulas that allow to perform numerical evaluation of the performance of the respective case. It also allowed to carry out the numerical experiments. The strategy of choosing the parameters to vary is that modern types of network (3G/LTE) have already got the necessary values of transmission power of both the base stations and mobile users, so the trasmit power was decided to keep as a constant. The same goes for the service threshold. As for thermal noise, then its value is also calculated, so it stays constant as well. So the densities and propagation exponents are left to be varied.
Based on the results of numerical experiments, a D2D enabled cellular network becomes useful when the signal propagation is seriously obstructed, say in the building of any kind. Also, based on the table results for figures 6 and 8, a D2D enabled cellular network yields better performance, when the ratio of the number of base stations to the number of mobile users is around 1/10,000.
\par The student Gulomov Saidkhuja would like to express his gratitude to academic advisor, prof. Naoto Miyoshi for kind support and help during the process of preparing this thesis. 
\par {\it Future Work.}
\\The future work may include the consideration of interference, a more sophisticated point processes (determinantal point process known to address the repulsion effect).
Also, the stationarity of the point process might be changed to the consideration of location changes over time.
\\A new layer of relay transmission can be added considering the developing of the project OneWeb that aims to provide Internet access from small satellites orbiting the Earth.
