\documentclass[10pt]{article}
\usepackage[utf8]{inputenc}
\usepackage{amsmath,amssymb}
\usepackage[english]{babel}
\tolerance 9000

\begin{document}
  \begin{center}
    {\bf Thesis Summary.}
  \end{center}
  Device-to-device (D2D) communication enables direct communication between nearby mobiles that improves spectrum utilization, overall throughput and energy efficiency. The thesis considers the performance of the D2D communication, i.e. connectivity, and compares with direct cellular link. The model assumes a noise-limited environment, thus interference between nodes is neglected and a key parameter of "Signal-to-Noise Ratio" is considered.
  \\ Utilizing methods of a stochastic geometry, a model of two independent homogeneous Poisson point processes was studied. The model incorporates such key parameters of the operating units, as transmission power, thermal noise, density of the units, propagation exponents and successful service thresholds. This key parameters are necessary later in the section of simulations for an accurate performance display.
  \\ The analysis of the performance of a direct cellular connection and a device-to-device relay connection is based upon the distance between the receiver, located at the origin of $\mathbb{R}^2$ and the nearest base station, whose position is denoted as $B_{nst}$.
  
\end{document}