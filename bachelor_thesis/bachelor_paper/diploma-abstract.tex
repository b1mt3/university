\newpage
\begin{abstract}
При нормальном распределении часто возникает задача определения вероятности того, что случайный вектор попадает
в первый октант. Например при нахождении вероятности ошибки декодирования при передаче кодового слова
по Гауссовскому каналу. В работе рассматривается частный случай $n=2$. Представлено решение задачи с помощью модели эллипсов рассеяния. Выведены формулы расчета вероятности посредством вычисления суммы площадей участков эллипсов. Проведена оценка погрешности при выборе данной модели. 
\end{abstract}
\selectlanguage{english}
\begin{abstract}
In a situation involving a gaussian distribution it happens to be a problem to calculate the probability that a random vector would fall onto the positive octant. For instance such problem exists when one tries to calculate the probability of a decoding error during the transmission of a code-word through the Gaussian channel. This paper considers an instance of 2-dimensional gaussian distribution. A solution is presented based on using the dispersion ellipses model. Formulas for calculation of the probabiltity by means of calculation of ellipse areas are carried out. A calculation error for the chosen model is estimated.
\end{abstract}
\selectlanguage{russian}