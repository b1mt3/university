\section{Основные определения и условные обозначения}
\subsection{Условные обозначения}
Аналитическое представление эллипсоида через матрицу ковариаций и вектор средних
$$\ell(\vec x)=(\vec x-\vec \mu)K^{-1}(\vec x-\vec \mu)^T$$
Площадь области $i-$го эллипсоида, ограниченного осями координат и частью графика эллипсоида
$$S_i = \{(x,y) \in R^{+}|\ell(x,y)\le R_i\}$$
Площадь слоя эллипсоида - области между $i-$ым и $(i+1)-$ым эллипсоидами
$$D_i = S_{i+1}\backslash S_i$$
Функция вычисления площади слоя эллипсоида
$$s_i = S(D_i)$$
Произвольная точка на слое эллипсоида
$$\xi_i=\{(x,y)|(x,y)\in D_i\}$$
$$\ell(\xi_i) = \zeta_i$$
Функция плотности распределения
$$f(x) = \frac{1}{\sqrt{4\pi^2detK}}e^{-\frac{1}{2}x}$$
\subsection{Вспомогательные определения и теоремы}
\parПреобразование Абеля
$$\sum_{k=1}^{N}a_kb_k = a_Nb_N-\sum_{k=1}^{N-1}B_k(a_{k+1}-a_k),$$
где $B_k = \sum_{i=0}^kb_i$
\par{\it Определение: $r_n$-тым остатком ряда $\sum_{k=1}^\infty a_k$ является ряд $\sum_{k=n+1}^\infty a_k$}
\par{\it Теорема(о среднем значении): Пусть $f(x)$ интегрируема в [a;b] и пусть во всем этом промежутке $m\le f(x)\le M$; тогда}
$$\int_a^b{f(x)dx = \mu(b - a)}$$
{\it где $m\le \mu\le M$}
\par{\it Теорема(интегральный признак Коши-МакЛорена): пусть ряд $\sum_{n=1}^\infty{a_n}$ имеет форму $$\sum_{n=1}^\infty{a_n}\equiv\sum_{n=1}^\infty{f(n)},$$ где $f(n)$ есть значение при $x=n$ некоторой функции $f(x)$, определенной для $x\ge 1$, непрерывной, положительной и монотонной. Тогда ряд $\sum_{n=1}^\infty{f(n)}$ сходится или расходится в зависимости от того, имеет ли функция $$F(x)=\int{f(x)dx}$$при $x\rightarrow+\infty$ конечный предел или нет.}
\subsection{Постановка задачи}

Имеется $2$--мерное нормальное распределение с вектором средних $\mu= (\mu_1,\mu_2)$
и матрицей ковариаций $K$, при этом заданы следующие условия на вектор средних и матрицу ковариаций:\\
$\mu = {(\mu_1,\mu_2\ |\ \mu_1 \le 0,\mu_2 \le 0)}$ и определитель матрицы ковариаций равен 1, т.е. переменные $x_1,x_2$ линейно зависимы.\\
Плотность определяется формулой 
$$p(x) = \frac{1}{\sqrt{(2\pi)^2\cdot \det(K)}}\cdot \exp(-\frac12
(x-\mu)\cdot K^{-1}\cdot(x-\mu)^\top).$$

Необходимо найти вероятность попадания случайного вектора в первый октант, т.е.
$$\R^2_+ = \{ (x_1,x_2) | x_1\ge0,x_2\ge0\}.$$

Указанная вероятность равна $$P = \iint\limits_{x\in\R^2_+}p(x)dx.$$
Задача сводится к тому, чтобы вычислить данный интеграл с точностью $\epsilon$