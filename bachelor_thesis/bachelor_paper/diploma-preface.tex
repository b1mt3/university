\newpage
\section{Введение}
\parЗадача вычисления вероятности попадания вектора в положительный октант сформировалась при дешифровке сигнала, проходящего по гауссовскому каналу, рекурсивным применением дешифрующей функции. Т.е. сигнал, состоящий из исходного сигнала и шума, при поступлении на дешифратор, многократно в нём обрабатывается. В результате, составляющие исходного сигнала, изначально независимые, становятся сильно зависимыми друг от друга. Матрица ковариаций при этом становится плохо обусловленной, а компоненты вектора средних - очень малыми.
\parВ работах [3], [4] и [5] проведены исследования в проблеме вычисления вероятности нормально распределенной величины. Также созданы алгоритмы, вычисляющий с наперёд заданной точностью величину вероятности.
\parОднако особенностью подхода, рассмотренного в данной работе, является именно начальные параметры вероятности - плохо обусловленная ковариационная матрица, собственные значения которой близки к нулю, и вектор средних с очень малыми компонентами. 
\parПодобного рода задача неправильно решается общеизвестными прикладными средствами, такими как пакет MatLAB.
\parВ качестве закона распределения было выбрано нормальное в силу того, что оно является самой распространенной вероятностной моделью в мире.
\parВ работе сначала представляется использование модели эллипсов рассеяния для подсчета вероятности. После показана реализация вычисления с помощью бесконечной суммы площадей слоёв эллипсоидов; проведено приближение до конечной суммы с последующей оценкой погрешности приближения.
\parВыражаю большую благодарность своему научному руководителю, Алексееву Дмитрию Владимировичу, за всевозможную поддержку, внимание и отзывчивость в процессе написания данной работы.